\label{2:experteninterview_results}
\begin{table}[h!]
\begin{tabular}{p{4cm}|p{6.7cm}|r}
Liste der Kategorien                & Beschreibung                                                                                   & Häufigkeit \\ \hline \hline
Trends                               &                                                                                                &                                \\
\hspace{5mm}Aufstiege                            & Eine graduelle Veränderung der Daten nach oben                                                 & 3                              \\
\hspace{5mm}Abstiege                             & Eine graduelle Veränderung der Daten nach unten                                                & 1                              \\
\hspace{5mm}Extremwerte                                & Ein signifikanter Spitzenwert in den Daten, negativ oder positiv                                                     & 4                              \\
\hspace{5mm}Offset/Drifts                        & Eine Nullpunktverschiebung in den Daten                                                        & 4                              \\
\hspace{5mm}Schwingungen                         & Wiederholte Schwankungen in den Daten                                                          & 6                              \\
\hspace{5mm}Abweichungen                         & Überschreitung der Daten aus dem zuvor festgelegten Wertebereich                                    & 4                            \\ \hline  
Daten                                & Allgemeine Aussagen zu den benutzten Daten                                                      & 5                              \\
\hspace{5mm}Extremwerte                          & Extreme Punkte in den Daten                                                               & 5                              \\
\hspace{5mm}Schädigungswerte                     & Die allgemeine Bezeichnung der Daten, die für die Schädigung einer Komponente berechnet werden  & 7                              \\ \hline  
Visualisierungen                     &                                                                                                &                                \\
\hspace{5mm}Zeitreihen                           & Daten über einen Zeitraum dargestellt                                                   & 3                              \\
\hspace{5mm}Zählmethoden     & In der Mechanik bereits existierende Arten von Darstellungen                                   & 2                              \\  \hline  
Plausibilisierung                    &                                                                                                &                                \\
\hspace{5mm}Defekte/Messfehler               & Defekte oder Messfehler der Sensoren in den Daten                                              & 7                              \\
\hspace{5mm}Fehler des Fahrers                   & Erkennung von individuellen Fehlern beim Fahren, Missuse oder sonstigen Inkonsistenzen                       & 3                              \\    \hline  
Dashboard                            & Aussagen zu der Unterseite Dashboard                                                           & 2                              \\   \hline  
Überwachung Dauerlauferprobung       & Aussagen zu der Unterseite Überwachung Dauerlauferprobung                                      & 8                              \\   \hline  
Vergleiche von Einsatzdaten          & Aussagen zu der Unterseite Vergleiche von Einsatzdaten                                         & 3                              \\\hline  
Data-Storytelling                    & Allgemeine Ideen zum Thema Data-Storytelling und im speziellen Vorschläge für Datengeschichten & 4                              \\\hline  
Wünsche                              & Allgemeine Wünsche für die Darstellungen                                                       & 4                            
\end{tabular}
\centering
\caption {Ergebnisse der Experteninterviews mit Kurzbeschreibungen und Häufigkeiten der Erwähnungen des Themas}
\end{table}
