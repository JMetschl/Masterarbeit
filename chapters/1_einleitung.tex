\chapter{Einleitung}
\begin{description}
\item[Motivation] \hfill \\\\
Täglich werden Menschen mit der Verarbeitung großer Datenmengen konfrontiert. Insbesondere Ingenieure und Maschinenbauer sind auf gute und leicht verständliche Visualisierungen angewiesen, um abstrakte Werte, wie beispielsweise auf Bauteile wirkende Kräfte, zu analysieren. Mit zunehmender Datenmenge wird es jedoch schwieriger, die wichtigen Merkmale visuell hervorzuheben. Dafür wird in dieser Arbeit das Data-Storytelling genutzt, welches Daten mittels geeigneter Graphen auf die relevanten Details reduziert. Das soll sowohl die Zeit minimieren, die von den Ingenieuren für die Auswertung der Daten aufgewendet wird, als auch den dabei anfallenden Mental Load.\\\\

\item[Ziel der Arbeit]\hfill \\\\
Das Hauptziel dieser Arbeit ist, das Data-Storytelling erfolgreich mithilfe des festgelegten Prozesses auf Graphen anzuwenden. Hierbei ist besonders wichtig, dass die fünf Schritte, die im Laufe der Arbeit genauer erklärt werden, im Prozess eingehalten werden. Als weitere Ziele lassen sich die Entwicklung der Trenderkennung mit JavaScript und die Entwicklung der Graphen mit plotly.js festlegen. Typischerweise werden Trenderkennungen eher mit Sprachen wie Python durchgeführt, aber für diese Arbeit soll die Trenderkennung mit TypeScript implementiert werden.

Am Ende der Arbeit soll eine fertige, funktionierende Webseite stehen. Dabei wird die Frage gestellt, inwiefern Data-Storytelling bei der Entwicklung von Graphen hilfreich sein kann und wie es bei Daten einsetzbar ist, die nicht einfach zu verstehen sind. Auch soll in dieser Arbeit evaluiert werden, ob der Prozess des Data-Storytellings sinnvoll ist. Zuletzt soll in Zusammenarbeit mit zukünftigen Benutzern bewertet werden, ob das Data-Storytelling auf Betriebsfestigkeitsdaten Vorteile gegenüber traditionellen Darstellungen bietet.\\\\

\item[Methode]\hfill \\\\
Da die im Rahmen dieser Arbeit entwickelte Webseite einen spezifischen Use Case behandelt, werden zuerst Experteninterviews mit mehreren Ingenieuren geführt. Dabei wird geklärt, welche Information aus den zu entwickelnden Graphen gewonnen werden soll und worauf die Experten bei der Auswertung besonders achten. \\
Auf Basis der Interviews wird Code entwickelt, der für die automatische Extraktion der relevanten Merkmale und Trends aus den Daten genutzt wird. Anschließend werden die von den Ingenieuren beschriebenen Graphen für die Webseite implementiert.\\
Diese Graphen werden mit Data-Storytelling erweitert, um die User-Experience und Geschwindigkeit zu verbessern. Bei dieser Erweiterung wird die Trenderkennung zum Hervorheben der Points of Interest genutzt. Nach dem Erweitern der Graphen mit Data-Storytelling wird die gesamte Webseite mit Angular implementiert. Hierfür wird ein bereits bestehendes Monorepository verwendet.\\
Abschließend wird in einem weiteren Experteninterview evaluiert, ob das Data-Storytelling in diesem Projekt erfolgreich war. Da es für die gleichen Daten bereits eine vorhandene Webseite mit Graphen gibt, die kein Data-Storytelling verwenden, bietet sich auch ein Vergleich der alten und neuen Graphen an.\\\\
\item[Gender-Hinweis]\hfill \\\\
Zur besseren Lesbarkeit wird in dieser Masterarbeit das generische Maskulinum verwendet. Die in dieser Arbeit verwendeten Personenbezeichnungen beziehen sich – sofern nicht anders kenntlich gemacht – auf alle Geschlechter.
\end{description}

% dieser besonders relevanten Punkte
% Website oder Webseite -> Duden: Website
% wie verbessert Data Storytelling die Geschwindigkeit
% mit Data Storytelling erweitert -> mit DS-Techniken erweitert ?
% wie wird evaluiert, ob das DS erfolgreich war? schnelleres Bearbeiten der gleichen Daten?