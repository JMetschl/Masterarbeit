\chapter{Einleitung}
\begin{description}
\item[Motivation] \hfill \\\\
Täglich arbeiten Menschen mit Datenmengen, die oft groß sind. Insbesondere Ingenieure und Maschinenbauer sind auf gute und leicht verständliche Visualisierungen angewiesen, um abstrakte Werte zu analysieren, wie Kräfte, die auf Bauteile wirken. Bei großen Datenmengen kann es aber schwierig sein, die wichtigen Informationen aus den Daten visuell hervorzuheben. Dafür wird in dieser Arbeit das Data-Storytelling genutzt, durch das viele Daten in Graphen auf die wichtigen Details reduziert werden sollen. Das soll sowohl die Zeitminimieren, die von den Ingenieuren aufgewendet wird, um die Daten auszuwerten, als auch den Mental Load, der dafür anfällt.\\\\

\item[Ziel der Arbeit]\hfill \\\\
Das Hauptziel dieser Arbeit ist, das Data-Storytelling erfolgreich mithilfe des festgelegten Prozesses auf Graphen anzuwenden. Hierbei ist besonders wichtig, dass die fünf Schritte, die im Laufe der Arbeit genauer erklärt werden, im Prozess eingehalten werden. Als weitere Ziele lassen sich die Entwicklung der Trenderkennung mit JavaScript und die Entwicklung der Graphen mit plotly.js festlegen. Typischerweise werden Trenderkennungen eher mit Sprachen wie Python durchgeführt, aber für diese Arbeit soll die Trenderkennung mit TypeScript implementiert werden. Am Ende der Arbeit soll eine fertige, funktionierende Webseite stehen. Dabei wird die Frage gestellt, inwiefern Data-Storytelling bei der Entwicklung von Graphen hilfreich sein kann und wie es bei Daten einsetzbar ist, die nicht einfach zu verstehen sind. Auch soll in dieser Arbeit gezeigt werden, ob der Prozess des Data-Storytellings sinnvoll ist. Zuletzt soll mithilfe exemplarischer Benutzer, der Ingenieure, in Erfahrung gebracht werden, ob das Data-Storytelling sinnvoll auf Betriebsfestigkeitsdaten anwendbar ist.\\\\

\item[Methode]\hfill \\\\
Da es sich bei der Webseite, die für diese Arbeit entwickelt wird, um eine Webseite mit einem spezifischen Usecase handelt, werden zuerst Experteninterviews mit mehreren Ingenieuren durchgeführt. Dabei ist es  wichtig, zu klären, welche Daten die Experten aus den zu entwickelnden Graphen gewinnen und worauf sie bei der Auswertung Wert legen. Auf Basis der Informationen aus den ausgewerteten Interviews wird Code entwickelt, der für die automatische Erkennung dieser besonders relevanten Punkte genutzt wird. Anschließend werden für die Webseite die Graphen, die auch von den Ingenieuren in den Interviews beschrieben wurden, implementiert. Diese Graphen werden mit Data-Storytelling erweitert, um die User-Experience und auch die Geschwindigkeit zu verbessern. Bei dieser Erweiterung wird ebenfalls die Trenderkennung genutzt, da das Hervorheben der Points of Interest sich besonders gut mit dem Data-Storytelling vereinbaren lässt. Nach dem Erweitern der Graphen mit Data-Storytelling wird die gesamte Webseite mit Angular implementiert. Hierfür wird ein bereits bestehendes Monorepository verwendet. Abschließend wird ein erneutes Experteninterview durchgeführt, das sich hauptsächlich mit der Frage beschäftigt, ob das Data-Storytelling in diesem Projekt erfolgreich war. Da es bereits eine vorhandene Webseite mit Graphen gibt, die ohne Data-Storytelling erstellt wurden, bietet es sich auch an, einen Vergleich der verschiedenen Graphen vorzunehmen.\\\\
\item[Gender-Hinweis]\hfill \\\\
Zur besseren Lesbarkeit wird in dieser Masterarbeit das generische Maskulinum verwendet. Die in dieser Arbeit verwendeten Personenbezeichnungen beziehen sich – sofern nicht anders kenntlich gemacht – auf alle Geschlechter.
\end{description}