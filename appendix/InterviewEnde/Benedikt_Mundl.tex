\section{Transkription Feedback Interview Benedikt Mundl} 
\label{appendix:interview_ende_mundl}
Interviewpartner: Diplomingenieur (FH) Benedikt Mundl, seit 15 Jahren in der Messdatenanalyse mit Fahrzeugen und Flugzeugen\\
Datum: 16. April 2024, um 9 Uhr\\
Dauer des Interviews: 7 Minuten\\

\begin{linenumbers}
\noindent
I: Hallo Herr Mundl, vielen Dank für deine Zeit. Also, es soll in dem Interview hauptsächlich geklärt werden, ob das Data Storytelling, was ich in der Arbeit ausgeführt habe, erfolgreich war, und ob es zur Verbesserung der Graphen beigetragen hat. Und dann noch ein paar persönliche Meinungen zum Data Storytelling. Ich habe wieder einen Interviewleitfaden mit einigen Fragen mitgebracht. Dann würde ich mal mit der ersten Frage starten. Hat das Data Storytelling dazu beigetragen, dass die Graphen leichter verständlich sind?\\\\
BM:\llabel{appendix:interview_ende_mundl:verstaendlich} Ja, auf jeden Fall, weil man nicht nur den Graphen als solchen sieht, sondern auch diese ganzen Zusatzinformationen dazu bekommt. Wie dieses geht es gerade irgendwo hoch, oder geht es runter, dann die Minima, Maxima. Da hat man halt alles auf einem Blick. \\\\
I: Gut, wird das Data Storytelling auch bei weiteren Projekten eingesetzt werden?\\\\
BM:\llabel{appendix:interview_ende_mundl:neue} In jedem Fall, weil wir uns natürlich freuen, wenn wir jetzt schon so eine neue, schöne Oberfläche haben, da wollen wir die natürlich auch weiter benutzen. Und das ist ja gerade für, vor allem, weniger geübte Anwender, was jetzt überhaupt nicht despektierlich klingen soll, aber für die ist es einfach hilfreich, wenn man auf so ein paar Kleinigkeiten hingewiesen wird. Und das sieht man dann vielleicht einfach ganz gut, und muss sich gar nicht so viele Gedanken machen, wie ich denn an die Informationen komme, die ich herauslesen will. Das heißt, wenn einem da quasi jemand, oder eben ein System, hilft, dann ist da ja sehr gut.\\\\
I: Würdest du sagen, dass durch das Data Storytelling mit den Graphen schneller gearbeitet werden kann?\\\\
BM:\llabel{appendix:interview_ende_mundl:geschwindigkeit} Ja, schneller ja, aber vor allem verständlicher würde ich sagen. Weil man, wie gesagt auf so ein paar Keyfacts, um es cool klingen zu lassen.\\\\
I: Ich habe es in der Arbeit Points of Interest genannt.\\\\
BM: Ja, auf die würd man ganz augenscheinlich hingewiesen. Und das ist natürlich ganz schön, weil dann muss man sich keine weiteren Gedanken machen. \\\\
I: Absolut. Und du persönlich, welche Vorteile werden durch das Data Storytelling erreicht?\\\\
BM:\llabel{appendix:interview_ende_mundl:leichter} Zum einen, wie gesagt, weniger versierte Nutzer finden sich ein bisschen einfacher oder schneller zurecht, oder man kann es ihnen zumindest deutlicher zeigen, das auf jeden Fall. Und gerade auch für einen selber, wenn es jetzt große, umfangreiche Mengen sind, die man sich da anschaut, hilft es natürlich immer, wenn man sich da Werkzeuge ein- oder ausblenden kann.\llabel{appendix:interview_ende_mundl:story} Und was natürlich auch noch ist, insgesamt, wird man selbst auf ein paar Sachen hingewiesen, also wenn ich jetzt sehe, der geht eigentlich immer nur nach oben, dann kann ich das ja einfach schon als Werkzeug anwenden, um eventuell Fehler in einer Messgröße zu detektieren. Weil, wenn ich sehe, der haut immer weiter ab, dann weiß ich, dass im \ac{DMS} ein Drift darauf sein kann. Das ist nicht gut, weil da soll eigentlich keiner darauf sein. Und das sehe ich gleich, ohne es anzuschauen, quasi. Ohne es zu betrachten, sehe ich es trotzdem.\\\\
I: Okay und gibt es Nachteile vom Data Storytelling?\\\\
BM:\llabel{appendix:interview_ende_mundl:nachteil} Der einzige Nachteil, der mir einfällt, ist quasi, wenn nicht hübsch umschrieben ist, was welche Einstellung macht. Also das hatten wir ja auch schon [in der Trenderkennung], die Texte, dass die eben einfach formuliert sein müssen. Weil sonst, wenn es zu kompliziert ist, und sich keiner, was darunter vorstellen kann, dann nimmt man es nicht, weil dann denkt sich jeder: WHAT? Und dann ist es schade, weil dann nimmt man es nicht und dann ist das sehr schade. Das ist das einzige, wo man glaube, ich auf den Punkt bringen muss, was kann dieser Knopf oder dieses Schieberchen. Was tut das? Dass sich eben der Anwender auch traut, das zu benutzen, oder halt, dass der den Sinn sieht, das zu benutzen. Dass man die Leute halbwegs schnell fängt, dass sie das auch benutzen.\\\\
I: Würdest du sagen, dass das Data Storytelling in dem Projekt erfolgreich war?\\\\
BM:\llabel{appendix:interview_ende_mundl:erfolgreich} Ja, auf jeden Fall, manche Themen, die durch das Data Storytelling initiiert wurden, kommen ja jetzt schon bei einem Folgeprojekt bei einem anderen Kunden als Kernaufgabe dazu, dass man erkennt, driftet das weg, oder ganz schnell Min-, Maxwerte hat, und ganz schnell sieht, sind die außerhalb einer Range. Da hat ja die Fortführung schon angefangen.\\\\
I: Okay, dann vielen Dank für das Interview und einen schönen Tag noch.
\end{linenumbers}