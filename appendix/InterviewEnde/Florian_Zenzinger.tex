\section{Transkription Feedback Interview Florian Zenzinger}
\label{appendix:interview_ende_zenzinger}
Interviewpartner: Diplomingenieur (FH) Maschinenbau Florian Zenzinger, arbeitet seit April 2016 in der Messdatenanalyse\\
Datum: 29. April 2024, um 14 Uhr\\
Dauer des Interviews: 2 Minuten\\

\begin{linenumbers}
\noindent
I: Hallo Herr Zenzinger, vielen Dank, dass du dir für das Interview Zeit nimmst. Also in diesem Interview soll geklärt werden, ob das Data Storytelling, das ausgeführt wurde, zur Verbesserung der Graphen beigetragen hat. Deswegen kommen wir gleich mal zur ersten Frage. Würdest du sagen, dass das Data Storytelling dazu beigetragen hat, dass die Graphen leichter verständlich sind?
\\\\
FZ: \llabel{appendix:interview_ende_zenzinger:leichter}Ja.
\\\\
I: Das ist schön.
\\\\
FZ: Was ich gesehen habe, auf jeden Fall, ja.
\\\\
I: Okay, wird das Data Storytelling auch bei weiteren Projekten eingesetzt werden und/oder nachgezogen werden?
\\\\
FZ:\llabel{appendix:interview_ende_zenzinger:neue} Da gehe ich davon aus.
\\\\
I: Würdest du sagen, dass durch das Data Storytelling mit den Graphen schneller gearbeitet werden kann?
\\\\
FZ:\llabel{appendix:interview_ende_zenzinger:schneller} Auf jeden Fall. Was ich gesehen habe, da mit dem ganzen Min, Max anzeigen, ja, da spart man sich Zeit. 
\\\\
I: Okay, und welche Vorteile sind dir durch das Data Storytelling aufgefallen?
\\\\
FZ:\llabel{appendix:interview_ende_zenzinger:schneller_2} Was ich gesehen habe, durch von deinen Graphen, ist es auf jeden Fall deutlich übersichtlicher. Und dadurch eben auch schneller, das heißt, wo man sonst jetzt in den Graphen herumsuchen musst, wo ist der Min-, Maxwert, und diese Steigungen, also eigentlich alles, wonach man sonst händisch suchen musste, übernimmt [die Trenderkennung] jetzt quasi und das wird einem direkt angezeigt.\\\\
I: Gut, und sind dir Nachteile durch das Data Storytelling aufgefallen?
\\\\
FZ: Ne.
\\\\
I: Und würdest du sagen, dass das Data Storytelling ein Erfolg war?
\\\\
FZ: So wie du es mir gezeigt hast, würde ich auf jeden Fall sagen, ja.
\\\\
I: Okay, sehr schön, dann danke ich dir für das Interview und deine Zeit.
\end{linenumbers} 