\section{Transkription Feedback Interview Michael Städele}
\label{appendix:interview_ende_staedele}
Interviewpartner: Diplomingenieur (FH) Michael Städele, seit 2008 in der Messdatenanalyse mit Fahrzeugen und Flugzeugen mit Betriebsfestigkeit\\
Datum: 19. April 2024, um 11:30 Uhr\\
Dauer des Interviews: 7 Minuten\\

\begin{linenumbers}
\noindent
I: Hallo Herr Städele, vielen Dank, dass du dir für das Interview Zeit nimmst.\\\\
MS: Hallo, kein Problem.\\\\
I: Also ich habe wieder einen Leitfaden mitgebracht, wie beim letzten Mal und würde einfach Fragen stellen und du kannst dann darauf antworten.\\\\
MS: Sehr schön.\\\\
I: Genau, also in dem Interview soll geklärt werden, ob das Data Storytelling, dass in meiner Arbeit durchgeführt wurde, zum Verbessern der Graphen beigetragen hat und ob es erfolgreich war. Also fangen wir gleich mal mit der ersten Frage an. Würdest du sagen, dass das Data Storytelling dazu beigetragen hat, dass die Graphen leichter verständlich sind?\\\\
MS:\llabel{appendix:interview_ende_staedele:story} Also auf alle Fälle. Man sieht in den Graphen eher das, worauf es ankommt. Gerade das mit diesen farblichen Hinterlegungen [in der Trenderkennung]. Da sieht man ja auf den ersten Blick gleich, welche Werte, wo interessant sind. Oder welches die interessanteren Werte sind. Man wird quasi optisch gleich darauf hingewiesen. Und auch in den Zeitreihen, die Markierungen dieser Extremwerte bringt eigentlich auch den Fokus dahin, auf die entsprechenden hohen Werte, auf die es dann eigentlich auch ankommt.\\\\
I: Sehr schön. Und planen wir das Data Storytelling in weiteren Projekten einzusetzen und/oder nachzuziehen?\\\\
MS:\llabel{appendix:interview_ende_staedele:neue} Auf alle Fälle. Gerade in den ganzen Datenbanken, planen wir auf jeden Fall es umzusetzen. Bei manchen Sachen ist noch die Frage, wie die Parameter entsprechend gesetzt werden müssen, damit man die Extremwerte, oder die Werte markiert kriegt, die man auch später haben will, aber das ist dann auch eine Sache, die dann in den Daten nochmal speziell geklärt werden muss. Aber generell, von den Funktionen, auf alle Fälle.\\\\
I: Gut, würdest du sagen, dass durch das Data Storytelling mit den Graphen schneller gearbeitet werden?\\\\
MS:\llabel{appendix:interview_ende_staedele:schneller} Ja, es kann, denke ich, sowohl schneller gearbeitet werden, also auch leichter, also leichter verständlich.\llabel{appendix:interview_ende_staedele:leichter} Einmal auch für Leute, die nicht ganz so in der Thematik drinstecken, werden sie eigentlich auf die entsprechenden Werte, die relevant sind, hingewiesen. Und wenn jetzt Fachleute daran sitzen, die können natürlich auch schneller damit arbeiten, weil sie auch auf diese Werte fokussiert werden und dann im ersten Blick schon sehen, bei Vergleichen, wie es aussieht, welche Belastungen, wo zu den entsprechenden hohen Werten geführt haben, die dann für das Fahrzeug schädigend sind.\\\\
I: Das ist jetzt vielleicht nochmal überschneidend, aber welche Vorteile wurden durch das Data Storytelling erreicht?\\\\
MS:\llabel{appendix:interview_ende_staedele:schneller_2} Also einmal natürlich, die entsprechende schnellere Möglichkeit damit zu arbeiten, \llabel{appendix:interview_ende_staedele:story_2} die Fokussierung auf die wichtigen Werte, die entsprechenden hohen Belastungen, und einfach durch diese farbliche Markierung, diesen Fokus auf die Werte zu lenken, die entsprechend wichtig sind.\\\\
I: Und gibt es Nachteile durch das Data Storytelling?\\\\
MS: Es sind mir aktuell keine bekannt.\\\\
I: Sehr schön.\\\\
MS:\llabel{appendix:interview_ende_staedele:nachteile} Nein, sind mir jetzt keine bekannt, also wie gesagt, das Einzige ist, dass man bei manchen Graphen, also gerade bei den Zeitreihen, schauen muss, welche Parameter, man wie so setzt, damit man vielleicht nicht jeden Peak markiert hat, sondern vielleicht über mehrere Graphen, dann die entsprechenden Extremwerte dann bewertet. Aber das sind ja alles Vorteile, also da gibt es ja keine Nachteile, das wäre dann nur noch eine Konfigurationssache.\\\\
I: Okay, dann sind wir auch schon bei der letzten Frage. Würdest du sagen, dass das Data Storytelling erfolgreich war?\\\\
MS:\llabel{appendix:interview_ende_staedele:erfolgreich} Na Logo, klar, war echt cool. Ist eine super Arbeit gewesen, da hast du einen richtig guten Job gemacht. Vielen Dank auf jeden Fall für die Arbeit. Weil es echt die Arbeit erleichtert und auch besser aussieht als davor.\\\\
I: Ja, das denke ich auch, wenn man nicht so erschlagen wird, von allen Daten.\\\\
MS: Ja, ganz genau.\\\\
I: Dann auf jeden Fall vielen Dank für das Interview und einen schönen Tag noch.
\end{linenumbers}