\section{Transkription Feedback Interview Jeele Böggemann}
\label{appendix:interview_ende_boeggemann}
Interviewpartner: Bachelor of Engineering Jeele Böggemann, seit 6 Jahren in der Messdatenanalyse mit Fahrzeugen und Flugzeugen\\
Datum: 12. April 2024, um 8:45 Uhr\\
Dauer des Interviews: 5 Minuten\\

\begin{linenumbers}
\noindent
I: Hallo, Herr Böggemann, vielen Dank, dass du dir für das Interview Zeit nimmst. Ich würde gerne zum Ende der Masterarbeit euch als Experten fragen, wie euer Feedback zum Data Storytelling ist. Dazu habe ich wieder einen Interviewleitfaden vorbereitet. Also es soll hauptsächlich geklärt werden, ob das Data Storytelling, das durchgeführt wurde, zur Verbesserung der Graphen beigetragen hat. Und ob es auch für so Daten, also so trockenere Fahrzeugdaten Sinn ergibt. Genau, dann würde ich erstmal fragen, wie deine Meinung zum Data Storytelling ist. Du hast das ja jetzt so ein bisschen mitbekommen, wie man das anwendet.\\\\
JB: Also seitdem du mir davon erzählt hast, fallen mir auf jeden Fall ständig Sachen ein, wo man sowas eventuell umsetzen könnte. Oder einbauen kann, um auch die bestehende Webseite vom Workflow her zu verbessern.\\\\
I: Optimal. Das ist natürlich sehr schön. Würdest du sagen, dass das Data Storytelling im Vergleich von den neuen Graphen und den neuen Graphen dazu beigetragen hat, dass die verständlicher sind? Oder dass man mit denen besser arbeiten kann?\\\\
JB:\llabel{appendix:interview_ende_boeggemann:schneller} Würde ich ja sagen, gerade wenn man sich auch viele Daten hintereinander anschaut. Weil, wenn ich jetzt wirklich sage, ich schaue mir nur einen Graphen an, und das war es dann, dann brauche ich es nicht unbedingt, weil dann habe ich auch die Zeit, oder auch noch den Kopf, um die Dinge schnell zu sehen. Aber wenn ich jetzt anfange, wirklich viel zu gucken, viel zu vergleichen, dann ist das definitiv ein Mehrwert.\\\\
I: Sehr gut, und du würdest auch unterstützen, dass das Data Storytelling noch bei weiteren Projekten eingesetzt wird, oder noch bei alten Projekten nachgezogen wird?\\\\
JB:\llabel{appendix:interview_ende_boeggemann:neue} Ja.\\\\
I: Gut, die letzte Frage ist, dann, ob durch das Data Storytelling die Graphen schneller bearbeitet werden können. Das hast du ja vorhin schon etwas angerissen.\\\\
JB:\llabel{appendix:interview_ende_boeggemann:schneller_2}  Würde ich so ja sagen, so wie es momentan umgesetzt ist, bei diesem Projekt, da müsste, ich mich allerdings auch mit der Funktion an sich nochmal genauer auseinandersetzten, um zu wissen, wie wende ich die vernünftig an. Beziehungsweise inwiefern beschleunigt mich das dann. Damit habe ich mich selber jetzt noch nicht so wirklich auseinandergesetzt.\\\\
I: Okay, und jetzt noch zur Trenderkennung. Würdest du sagen, dass man die noch erweitern und/oder verbessern kann?\\\\
JB: Das ist schwierig zu sagen, weil ich mit der Projektseite noch nicht wirklich gearbeitet habe. Also ich kenne die auch einfach bloß aus den Präsentationen und Demos, die wir gemacht haben. Um da eine vernünftige Antwort zu geben, müsste ich mir das selber auch mit mehr Verstand angucken.\\\\
I: Verstehe, okay. Das sind ja hauptsächlich so Auf- und Abstiege, Minima und Maxima, Extremwerte.\\\\
JB: Ja jetzt gerade in Bezug darauf, wie man die verbessern könnte, da kann ich keine Aussage zu treffen.\\\\
I: Würdest du sagen, dass das Data Storytelling in dem Projekt erfolgreich war?\\\\
JB:\llabel{appendix:interview_ende_boeggemann:erfolgreich} Das würde ich sagen, ja.\\\\
I: Sehr gut, dann vielen Dank für das Interview!
\end{linenumbers}