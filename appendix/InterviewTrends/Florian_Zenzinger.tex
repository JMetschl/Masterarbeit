\section{Transkription Interview Florian Zenzinger}
\label{appendix:interview_trends_zenzinger}
Interviewpartner: Diplomingenieur (FH) Maschinenbau Florian Zenzinger, arbeitet seit April 2016 in der Messdatenanalyse\\
Datum: 16. November 2023, um 15 Uhr\\
Dauer des Interviews: 12 Minuten\\

\begin{linenumbers}
\noindent
I: Hallo Florian, erstmal vielen Dank, dass du dir Zeit für das Interview nimmst. Erstmal möchte ich dir das Thema meiner Masterarbeit erklären. Und zwar, geht es bei meiner Masterarbeit um das neue Kundenprojekt, bzw. das neue Frontend für das Projekt. Und ich möchte Graphen mit Data Storytelling darstellen. Hast du schon mal etwas von Data Storytelling gehört?\\\\
FZ: Ich überlege gerade, ich habe jetzt keine explizite Definition dazu im Kopf.\\\\
I: Okay, ja, also es ist einfach eine Art wie man Graphen darstellt, dass man so ein bisschen eine Geschichte aus den Daten herstellt und dann auch interessante Punkte in den Graphen highlightet. Also was, was jetzt für dich als Ingenieur, oder als jemand der sich das anschaut und tatsächlich weiß, was die Daten bedeuten, was der interessant findet. Und dafür habe ich die Forschungsfrage, welche Trends man in den Daten erkennen kann, um die in den Daten direkt darzustellen.\\\\
FZ: Im Sinne jetzt von sämtlichen Daten, die wir als Ingenieur anschauen oder auf spezielle Daten bezogen?\\\\
I: Im Allgemeinen alle Daten, die wir bekommen und die dann halt irgendwie dargestellt werden sollen.\\\\
FZ: Okay, da kann ich so ein bisschen allgemein erzählen, weil in dem Kundenprojekt bin ich selber nicht wirklich involviert.\\\\
I: Ja, genau, das hat mir Michael Städele schon erzählt, aber er hat auch gesagt, dass der Prozess immer gleich ist, und worauf man achtet auch.\\\\
FZ:\llabel{appendix:interview_trends_zenzinger:plausibilisierung_1} Genau, also wenn man die Kette mal so ein bisschen durchspielt, gibt es verschiedene Themen. Wenn man mal ganz zu Beginn sieht, ob die Daten überhaupt sinnvoll sind. Nennen wir das mal so.\llabel{appendix:interview_trends_zenzinger:peaks_1}  Wonach wir da dann erstmal suchen, ist, ob du irgendwelche Spikes zum Beispiel drinnen hast.\llabel{appendix:interview_trends_zenzinger:defekte_1} Also wirklich Fehleranalysen, dass du siehst, da ist vielleicht irgendwo ein Messwert hochgerutscht, so nach dem Motto Fehler von der Messtechnik. Dass der einen Wert anzeigt, der in der Realität einfach gar nicht möglich ist. Also wo dann vielleicht der Sensor einen Fehler hat. Nennen wir das mal so, wenn wir da mal von Anfang an anfangen. Oder, dass du Fehler drinnen hast, wie er zeichnet auch einfach nicht richtig auf.\llabel{appendix:interview_trends_zenzinger:abweichungen_1} Das da einfach leere Stücke in den Daten sind. Dementsprechend dann auch in der Darstellung siehst, da ist ein Loch. Da sollte kein Loch sein, aber da ist einfach ein Loch und dementsprechend fehlen da Daten. Das wäre dann mal, sodass wo man sagt, das kann nicht plausibel sein. Auch, ob die Werte selbst plausibel sind. Da gibt es zum Beispiel einen Längslenker und der sollte in der Realität normal 10 Kilonewton anzeigen, du siehst aber in den Messdaten, das sind 100 oder 1000 sogar, da muss also auch ein Fehler drinnen sein. Also das ist so, wenn man ganz vorne an der Kette anfängt, wo erstmal geprüft wird, ob die Daten so sein können. Und dann im Sinne von Analysen, ist die Frage, ob das spezifisch auf das Kundenprojekt ist oder generell auf Messdaten, was einen da so interessieren könnte. \\\\   
I: Ich glaube eher für das Projekt selber, weil wir da ja auch schon Unterseiten festgelegt haben und was die darstellen sollen. Also was in den Graphen dargestellt wird. Ich kann dir das ja einfach mal auflisten und du kannst mir deinen Input geben.\\\\
FZ: Ja, genau, da fehlt mir sonst ein bisschen der Kontext, was da gebraucht wird.\\\\
I: Ja, auch wenn du jetzt gar nicht damit arbeitest, bist du ja trotzdem vom Fach und weißt, was spannend wäre.\\\\
FZ: So ungefähr, aber das kannst du mir ja gerne mal darstellen.\\\\
I: Ja gerne, das erste ist eine Seite für die Datenbasis, die kannst du dir ähnlich vorstellen wie bei Porsche. Also wo man dann auch eine Übersicht über die Fahrzeuge hat und Karten, wo die Einsätze sind. Genau, da gibt es ja eigentlich noch nicht wirklich einen Graphen, der dargestellt wird. Dann gibt es als Nächstes eine Seite die Überwachung Dauerlauferprobung heißt, wo dann eine Dauerlauferprobung von den Fahrzeugen auf dem Testgelände analysiert wird. Die halt immer gleich ist, aber X Umfahrten zeigt.\\\\
FZ: Ja, genau, das ist so ein Standard. Ich weiß nicht, ob du so etwas schon mal gesehen hast.\\\\
I: So eine Teststrecke, nein, habe ich nicht.\\\\
FZ:\llabel{appendix:interview_trends_zenzinger:dauerlauferprobung_1} Das kannst du dir vorstellen wie so ein relativ grober Go-Kart-Kurs. So in der Art. Ein bisschen größer natürlich. Und da fahren die einfach den ganzen Tag die gleiche Strecke auf und ab.\\\\
I: Ah okay, und die letzte ist dann ein Vergleich von Einsatzdaten, wo man dann die, die auf der Dauerlauferprobung erfahren hat, mit realen Einsätzen vergleichen kann.\\\\
FZ: Okay, und jetzt in dem Sinne, was mich als Ingenieur da interessieren würde, in diesen Daten.\\\\
I: Genau.\\\\
FZ:\llabel{appendix:interview_trends_zenzinger:dashboard_1} Gut, was man da natürlich klassisch macht, ist, nachdem man sich die Datenbasis und die Einsatzorte angeschaut hat, ist nachzuschauen, wie viele Kilometer die wo im Einsatz sind. Zum Beispiel.\llabel{appendix:interview_trends_zenzinger:storytelling_1} Da gibt es ja auch im Data Storytelling alle möglichen Arten, das darzustellen, eben entweder so Hotspots bilden, wo man sieht, in München sind sehr viele Kilometer abgebildet, weil die Fahrzeuge halt viel in der Stadt sind. Und auf dem Land fahren sie eher weniger herum. Oder halt genau andersherum. Also damit man schon mal weiß welche Einsatzgebiete die zum Beispiel haben. Und wie die aufgebaut sind. So und so viel Prozent sind sie da unterwegs und so und so viel Prozent woanders. Was dann an den Daten selber interessant ist, ist auch ebenfalls ähnlich wie auch bei Porsche in der Lastdatenbank haben,\llabel{appendix:interview_trends_zenzinger:extremwerte_1} sind die Maximalwerte, also wie hoch sind unsere Maximalwerte, die da auftreten, und natürlich dann auch noch die Verknüpfung, wo findet das statt. Und wenn es dann weiter geht, da kommen wir dann aber schon wieder in so Berechnungsdarstellungen.\llabel{appendix:interview_trends_zenzinger:schädigungswerte_1} Wenn man aus den ganzen Daten dann die sogenannte Schädigung berechnet, die ja letzten Endes, angibt, wie schnell Bauteile ermüden. Oder ein Indikator dafür ist, wie lang die Lebensdauer von diesen Bauteilen ist. Dann kommen wir auch wieder in das gleiche Gebiet rein, wo mich interessiert, welche Strecken haben die höchsten Schädigungen, wo finden die statt, auch welches Fahrzeug ist es dann.\llabel{appendix:interview_trends_zenzinger:schwingungen_1} Genau, so das gleiche Spiel kann man auch noch mit Frequenzen machen.\llabel{appendix:interview_trends_zenzinger:zeitreihen_1} Also wenn du dir jetzt vorstellst, in so einem Zeitschrieb, da sieht man an gewissen Stellen, da ist sehr viel los.\llabel{appendix:interview_trends_zenzinger:schwingungen_2} Also viele Schwingungen. Dass du auch siehst, in welchem Frequenzbereich das ist. Also schwingt das eher in der Eigenfrequenz oder ist eine angeregte Frequenz. Das sind dann alles relevante Informationen für gewisse Erklärungen.\\\\
I: Sowas ist dann ja wahrscheinlich auch ein Trend, der interessant wäre, wenn die Frequenz einfach immer und immer schneller wird, oder?\\\\
FZ:\llabel{appendix:interview_trends_zenzinger:schwingungen_3}\llabel{appendix:interview_trends_zenzinger:zeitreihen_2} Ja genau, und auch generell, in welcher Frequenz, weil auch jedes Bauteil, hat eine gewisse Eigenfrequenz. Das kann man sich vorstellen wie, wenn man klassisch mit dem Lineal am Schreibtisch selber schwingen lässt. Also einmal antippen, dann schwingt das in einer Eigenfrequenz vor sich hin. Und wenn man jetzt aber eine Strecke hast, wo all diese Dinge in ihrer Eigenfrequenz schwingen sollten, Achse, Rahmen, usw. dann ist das ziemlich schlecht. Das sollte man immer möglichst vermeiden, weil sich das über das ganze Fahrzeug überträgt. Das heißt sowas eben, dass man also wirklich sieht welche Frequenzen und auch im Zeitschrieb solche Ballungen hast, wo man sieht, da sind sehr viele Schwingungen an einem Ort. Das ist immer das, wo man dann hinschaut, um zu sehen, was da überhaupt passiert. Warum passiert das?\llabel{appendix:interview_trends_zenzinger:schwingungen_4} Welche Frequenz ist das? Was sind die Maximalwerte? \\\\
I: Kann man jetzt, bei den Frequenzen einfach einen Durchschnittswert berechnen und dann anhand von dem Durchschnitt erkennen, wann es viel zu hoch ist oder ist das zu einfach?\\\\
FZ:\llabel{appendix:interview_trends_zenzinger:schwingungen_5} Nein, also so in der Art geht das schon, entweder man macht es so wie du sagst, man zoomt einfach rein und schaut in welcher Frequenz ist das. Die Berechnung geht dann durch das Frequenzfenster durch und da wo die maximalen Schwingungen sind, oder die maximalen Werte, das ist dann eher das, was dich interessiert.\llabel{appendix:interview_trends_zenzinger:daten_1} Und eine andere Darstellungsmethode, die es da noch gibt, ist das sogenannte \ac{PSD}. Und das gibt letzten Endes an, über die ganze Datei, in welchem Frequenzbereich der meiste Energieeintrag ist. Also das ist eine andere Darstellung, aber die besagt, dann Achtung bei der Strecke X ist bei 30 Hertz am meisten passiert. Da schwingt das Teil am meisten auf und ab. Aber so in der Art gibt es das. Und wenn du es jetzt genau bei so einem maximal Fenster siehst, da passiert sehr viel, dann kannst du theoretisch schon reinzoomen und das über den Mittelwert machen. Das geht meistens durch so ein kleines Frequenzfenster durch.\\\\
I: Ja, der Jeele hat auch so ein paar Sachen über Offsets und so erzählt.\\\\
FZ:\llabel{appendix:interview_trends_zenzinger:offset_1} Das, ja, stimmt, das ist auch interessant. Also Offsets ist ebenfalls so ein Thema, wenn man irgendwo sieht, dass im Zeitschrieb auf einmal das Signal auf irgendeiner Höhe hängenbleibt. Also, es geht nicht wieder auf den ursprünglichen Wert zurück. Also nicht mehr auf die Nulllinie, sondern es bleibt auf dem Wert 5 hängen. Das ist auch immer interessant, weil entweder heißt es, dass du irgendwo was plastisch verformt hast, also es hat sich einfach was verbogen. Oder vielleicht ist auch der Sensor kaputt, also das sind so die zwei üblichen Sachen, wenn es so einen Sprung nach oben gibt. Und die ganzen Werte sind dann entsprechend weiter nach oben gesetzt, oder nach unten gesetzt.\\\\
I: Okay, das wäre ja aber auch etwas, was ich mathematisch erkennen könnte, um es dann darzustellen. Dann irgendwie so eine Art Warnung, Achtung hier kann es sein, dass ein Offset vorhanden ist.\\\\
FZ:\llabel{appendix:interview_trends_zenzinger:offset_2} Genau, absolut, absolut. Also das ist so ein Klassiker, irgendwie einen Mittelwert über das Ganze, oder einzelne Sektoren, bilden und dann siehst, du in dem Mittelwert, von dem Sektor ist auf einmal so und so viel höher, dann kann man schon sagen, das ist ziemlich wahrscheinlich, dass da ein Offset vorgekommen ist. Gerade zum Beispiel bei Beschleunigungen oder eine lange Kurvenfahrt stattfindet, wo es alles in eine Richtung zieht.\\\\
I: Alles klar, dann hast du mir auch schon alle meine Fragen beantwortet, vielen Dank.\\\\
FZ: Ja gerne, falls du sonst noch Fragen hast, melde dich gerne wieder. 
\end{linenumbers}