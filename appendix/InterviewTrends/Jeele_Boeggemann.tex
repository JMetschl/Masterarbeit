\section{Transkription Interview Jeele Böggemann}
\label{appendix:interview_trends_boeggemann}
Interviewpartner: Bachelor of Engineering Jeele Böggemann, seit 6 Jahren in der Messdatenanalyse mit Fahrzeugen und Flugzeugen\\
Datum: 13. November 2023, um 13 Uhr\\
Dauer des Interviews: 15 Minuten\\

\begin{linenumbers}
\noindent
I: Hallo Jeele, danke dass du dir Zeit für das Interview nimmst. Erstmal möchte ich dir gerne das Thema meiner Masterarbeit erklären. Es geht darum das Frontend des neuen Kundenprojekts mit Data Storytelling zu bauen. Hast du Data Storytelling schon mal gehört?\\\\
JB: Nein, das sagt mir noch nichts.\\\\
I: Okay, das ist eine Art, wie man Graphen erstellt. Dass man quasi eine Geschichte im Graphen erzählt und mein Ziel ist, damit eigentlich die ganze Bedienung der Seite zu erleichtern. Also automatisch Auf- und Abstiege erkennt und die dann extra hervorhebt. Genau, deswegen wäre es für mich sehr spannend, worauf du achtest, wenn du die Daten über das Frontend anschaust.\\\\
JB: Ah, ich verstehe.\\\\
I: Gut, dann starten wir gleich mal. Bei den Daten, welche Daten werden da genau von den Sensoren geliefert?\\\\
JB: \llabel{appendix:interview_trends_boeggemann:daten_1}Okay, da haben wir \ac{DMS}, mehrere, an den Federbeinaufhängungen, da sind an der Vorderachse und an der Hinterachse jeweils vier, schräg gegenüber voneinander. Dann sind an der Spurstange, an dem Lenkgestänge sind noch welche appliziert. Generell, am Fahrwerk sind \ac{DMS}, am Fahrersitz wird noch die Beschleunigung aufgezeichnet. In alle drei Richtungen, plus die Ableitung dazu, also die Drehraten um die Achsen.\\\\
I: Okay, und wenn du dir das jetzt mal so vorstellst, wie das im Porsche Frontend aussieht. Worauf achtest du, wenn du dir die Daten anschaust?\\\\
JB:\llabel{appendix:interview_trends_boeggemann:abweichungen_1}\llabel{appendix:interview_trends_boeggemann:extremwerte_1} Ich schaue mir als allererstes Mal die Extremwerte an, um zu sehen, ob da Ausreißer drin sind.\llabel{appendix:interview_trends_boeggemann:abweichungen_2}\llabel{appendix:interview_trends_boeggemann:defekte_1} Wenn mir da ein Wert auffällt, weil er aus dem Rahmen fliegt, dann schaue ich mir die dazugehörige Zeitreihe an, um zu plausibilisieren, ob der Wert echt ist, oder ob das ein Messfehler ist.\llabel{appendix:interview_trends_boeggemann:abweichungen_3}\llabel{appendix:interview_trends_boeggemann:schädigungswerte_1} Sowas Ähnliches mache ich dann mit den Schädigungen auch, also ich prüfe, ob die alle in einem ähnlichen Bereich sind, also entweder aus Erfahrung oder aus ähnlichen Fahrzeugen. Weil mittlerweile gibt es auch eine recht große Basis an Altmessungen. Die auch immer zum Gegenchecken verwendet werden kann. Und wenn dann irgendwo etwas Auffälliges ist, heißt die Werte sind zu hoch oder zu niedrig, dann geht es in den vertiefteren Vergleich zu den alten Fahrzeugen oder es wird geschaut, wie wurden denn überhaupt auf der Strecke gefahren. Ab und zu schaue ich mir auch Videos an von der Rennstrecke, um einfach die Kurven zu sehen. Um abschätzen zu können, was in der Kurve passiert. Kann das alles sein? Und dann ist es je nachdem was für eine Auswertemethode benutzt wird. Wenn jetzt konventionell ausgewertet wird, also nur ein Outing genommen wird und dass das irgendwohin extrapoliert wird, dann bin ich eigentlich fertig mit der Auswertung und wenn Rennsportmethode ausgewertet werden soll, dann werden noch die Streuungen beobachtet. Da gibt es dann von dem Kunden eine Vorgabe, wie hoch die Streuung sein darf und wenn ich über den Wert komme, dann werden nochmal einzelne Runden vergleichen, um zu entscheiden, ob die Schädigung nach oben, also das wird dann etwas konservativer ausgewertet oder festlege, dass die hohen Streuungen eher Ausreißer sind und die Schädigung weiter nach unten schiebe. \\\\
I: Vielen Dank, und gibt es dann auch so Trends oder sowas auf die du achtest bei den Daten selber?\\\\
JB: Eigentlich weniger.\\\\
I: Also das sind dann wirklich diese Ausreißer, die dich interessieren? Sind die dann spannend oder eher unschön, weil man sie vermeiden möchte?\\\\
JB:\llabel{appendix:interview_trends_boeggemann:daten_2}\llabel{appendix:interview_trends_boeggemann:fahrer_1} Kommt ganz darauf an. Kommt immer darauf an, wie das Fahrzeug ausgelegt ist, auch wo mein Target liegt. Also quasi der Wert, mit dem das Fahrzeug abgeprüft worden ist. Wenn ich da irgendwo darunter liege, ist die Messung in Ordnung.\llabel{appendix:interview_trends_boeggemann:abweichungen_4} Wenn ich über dem Target darüber liege, muss man auch nochmal genauer in die Daten gehen.\llabel{appendix:interview_trends_boeggemann:fehler_1} Und dann wird irgendwo auch eine Entscheidung getroffen, ob die Messung repräsentativ ist, ob das Fahrzeug das abkönnen muss, was da in der Fahrt passiert. Oder geht das eher Richtung Missuse.\\\\
I: Und die Targets werden dann vom Kunden festgelegt, oder?\\\\
JB:\llabel{appendix:interview_trends_boeggemann:daten_3} Genau, also normalerweise sind die Targets schon fertig, bevor wir mit dem Testen anfangen. Das ist eher eine Ausnahme, dass man das im Nachhinein festlegt. \\\\
I: Also wir haben ja für das Frontend schon Unterseiten festgelegt für das Frontend. Da haben wir einmal die Datenbasis, also quasi ein Dashboard, dass man schonmal weiß, welche Fahrzeuge es gibt und wie viel die fahren. Da gibt es ja noch keine Visualisierung. Gebe es da irgendwas was du als Visualisierung spannend fändest als Benutzer?\\\\
JB: Was jetzt anders als bei Porsche ist?\\\\
I: Darf anders als bei Porsche sein. Es gibt dann auch noch die anderen Unterseiten, also die nächste ist die Überwachung Dauerlauferprobung.\\\\
JB:\llabel{appendix:interview_trends_boeggemann:dauerlauferprobung_1} Ja, okay, also bei dieser Auswerteseite, die ist ja mehr oder weniger so gewachsen, wie ich oder wir die gerne hätte. Daher fällt mir da so spontan nichts ein, was ich da gerne anders hätte. Weil die ja aus der Erfahrung raus, wie wir Auswerten entstanden ist. Es gibt halt dann den Schnellzugriff auf die Einzelrunden, um halt schnell einen Extremwert zu plausibilisieren. Was da noch interessant wäre, wenn man zwei Runden miteinander vergleichen könnte.\\\\
I: Ja gut, also bei dem Kunden ist es ja so, dass man die Dauerlauferprobung hat und dann als zweite Seite den Vergleich von Einsatzdaten. Also das ist dann schon als eigene Seite mitberücksichtigt. \\\\
JB:\llabel{appendix:interview_trends_boeggemann:zaehlmethoden_1}  Die Kollektive kann man auch direkt sehen. Nein, da fällt mir echt nichts ein, was da irgendwie noch gebraucht wird. Ich komme ja da eigentlich recht schnell an alle Informationen ran, die ich brauche.\\\\
I: Okay, an sich möchte ich ja mit der Masterarbeit die Graphen so weiterentwickeln, dass die euch schon Arbeit abnehmen, dass ich als die Daten bekomme und schon erkenne, okay, hier passiert etwas Komisches und dann kann ich das direkt farbig darstellen und die anderen Daten, bei denen nichts Gruseliges passiert, werden als Kontext dargestellt.\\\\
JB:\llabel{appendix:interview_trends_boeggemann:wunsch_1} Oh okay, also was da vielleicht hilfreich sein könnte, wäre, wenn man die Balkendiagramme noch sortieren könnte. Von groß nach klein.\\\\
I: Also einfach von den Werten dann?\\\\
JB: Genau.\\\\
I: Ich bin mir noch gar nicht sicher, ob das überhaupt Balkendiagramme bleiben, vielleicht wird das auch was ganz was Neues. Ich fange bei den Graphen von null an, also, falls du irgendwelche Ideen oder Wünsche hast, was als Benutzer cool wäre, was dir die Arbeit erleichtern würde, kannst du mir jederzeit Bescheid geben.\\\\
JB:\llabel{appendix:interview_trends_boeggemann:wunsch_2} Ja, vielleicht, was helfen könnte, da müsste ich aber auch nochmal mehr darüber nachdenken, wären mehr Hovereffekte.\\\\
I: Auf jeden Fall.\\\\
JB:\llabel{appendix:interview_trends_boeggemann:extremwerte_2}\llabel{appendix:interview_trends_boeggemann:wunsch_3} Das, wenn es quasi über eine Runde geht, man dann einen Extremwert hat, also das haben wir auch schon häufiger gemacht in anderen Auswertungen, wie oft kommt dieser Extremwert oder halt nahe diesem Extremwert das Ganze vor. Sodass man nicht erst ins Kollektiv schauen muss, sondern irgendwo mit der Maus darüber hovern, und dann sieht man, dass es da beispielsweise 20 Kilonewton sind und zwei Prozent weniger kam auch noch so oft in den Daten vor.\\\\
I: Ja zum Beispiel sowas. \\\\
JB:\llabel{appendix:interview_trends_boeggemann:abweichungen_5} Dann könnte man auch leichter entscheiden, ob das ein Ausreißer ist, oder ob der [Fahrer] die Runde halt einfach so gefahren ist.\\\\
I: Das ist genau so etwas, dass ich suche, so Trends um die Ausreißer zu erkennen.\\\\
JB:\llabel{appendix:interview_trends_boeggemann:offset_1} Dann gibt es ja noch mehrere so statistische Kennwerte, die man aus den Daten auslesen kann, wie zum Beispiel ein Median, das wäre vielleicht interessant, um zu erkennen, ob das ganz über die Zeit weg gedriftet ist, oder ob da ein Offset eingegeben ist. Das wäre dann Richtung plausibilisieren, und gar nicht mehr so Richtung analysieren. Aber ansonsten müsste ich da echt noch länger drüber nachdenken. \\\\
I: Alles gut, das hat mir schon sehr weitergeholfen. Du kannst dich ja auch gerne jederzeit melden, wenn dir in der Benutzung etwas auffällt. Dann danke ich dir auf jeden Fall sehr für deine Zeit und wünsche dir noch einen schönen Tag.\\\\
JB: Danke gleichfalls.
\end{linenumbers}