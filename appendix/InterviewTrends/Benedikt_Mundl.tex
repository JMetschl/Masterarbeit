\section{Transkription Interview Benedikt Mundl}
\label{appendix:interview_trends_mundl}
Interviewpartner: Diplomingenieur (FH) Benedikt Mundl, seit 15 Jahren in der Messdatenanalyse mit Fahrzeugen und Flugzeugen\\
Datum: 6. November 2023, um 9 Uhr\\
Dauer des Interviews: 24 Minuten\\

\begin{linenumbers}
\noindent
I: Hallo Herr Mundl, vielen Dank, dass du dir für das Interview Zeit nimmst.\\\\
BM: Gar kein Problem, ich beantworte deine Fragen gerne.\\\\
I: Wie du schon weißt, besteht meine Masterarbeit aus dem Frontend der Webseite mit Verwendung von Data Storytelling und zusätzlich noch dem Definieren von mathematischen Regeln, mit denen ich Trends in unseren Daten erkennen kann, um die besser darzustellen.\\
Das heißt für mich ist es sehr interessant zu wissen, worauf ihr genau achtet, wenn ihr die Daten auswertet, wenn ihr euch die anschaut, ob es da irgendwelche Auf- und Abstiege gibt oder im Allgemeinen wie die Daten dargestellt werden sollen.\\\\
BM:\llabel{appendix:interview_trends_mundl:defekte_1} Ja, das heißt, es geht auch um so Dinge wie Messfehler oder ähnliches?\\\\
I: Zum Beispiel, ja.\\\\
BM: Okay, das ist eine sehr umfangreiche Frage.\\\\
I: Es ist ja auch eine Masterarbeit.\\\\
BM:\llabel{appendix:interview_trends_mundl:extremwerte_1} Also gut, wenn ich Daten zum ersten Mal sehe, dann schaue ich als Erstes auf den Wertebereich, heißt, entspricht, das, was da [von den Sensoren] ankommt, ungefähr dem, was zu vermuten ist. Was zu vermuten ist, das weiß man aus Erfahrung. Da kann man sich nicht genau auf eine Zahl festlegen, man muss halt wissen, okay, das sollte in so und so einem Bereich liegen.\llabel{appendix:interview_trends_mundl:defekte_2} Wenn jetzt bei einer Beschleunigung auf einmal fünf Pausen herauskommen, dann weiß ich schon, das ist Quatsch.\llabel{appendix:interview_trends_mundl:plausibilisierung_1} So, wenn die Schädigung dann irgendwo 5 g anzeigt, dann ist das für den ersten Eindruck in Ordnung, kann natürlich auch zu groß oder zu klein sein, aber für den ersten Moment ist die Zahl plausibel.\\ \llabel{appendix:interview_trends_mundl:peaks_1}Als Nächstes schaue ich mir an, ob ich auf den ersten Blick bereits einige Peaks sehe, irgendwelche Ausreißer, die deutlich über diesem gewünschten, oder anvisiertem Wertebereich liegt. Das sind dann immer mal Einzelwerte, die auf einmal irgendwo hinspringen, die sind dann durch beispielsweise defekte Sensoren bei der Datenerfassung selber aufgetreten. Das kann aus verschiedenen Gründen vorkommen, aber die Gründe sind ja erstmal egal, wir wollen diese Werte ja einfach nicht haben, in der Auswertung.\\ Und da fallen diese Werte dann relativ schnell auf, wenn man sich grafische Zeitschriebe ansieht, wenn man die Daten nur als reine Zahlen ansieht, sind diese Werte ziemlich schwer zu finden.\llabel{appendix:interview_trends_mundl:offset_1} Als Nächstes beschäftige ich mich mit Drifts. Drifts bedeutet bei uns, ein Kanal fängt bei einem Ausgangswert an, idealerweise 0, und verändert sich ganz langsam und stetig über das komplette Zeitsignal nach oben oder nach unten und endet dann bei einem neuen Offsetwert. Das passiert relativ häufig bei beispielsweise Temperaturwerten. Und dann schaue ich auf so Defaultwerte wie beispielsweise eine Samplingrate mit einem Hertzwert, sind die Einheiten korrekt, sind die Namen in Ordnung, diese Werte kann man auch als Metawerte bezeichnen. Ist das ungefähr die Antwort auf die Frage?\\\\
I: Ja, das hilft mir sehr weiter. Das war eine sehr ausführliche Antwort. Ich muss erstmal euren Prozess verstehen, um später die Darstellung im Data Storytelling so gutzumachen, dass euch dadurch Arbeit abgenommen werden kann.\\\\
BM: Ich habe mir mal ein Heft mit den verschiedenen Fällen, die bei der Analyse vorkommen können, angefertigt, das kann ich dir gerne zukommen lassen.\\\\
I: Ja, sehr gerne, das wäre sehr hilfreich!\\\\
BM: Kein Problem.\\\\
I: Wir haben ja bereits Unterseiten für unsere Webseite festgelegt. Worauf achtest du, wenn du die Daten in Bezug auf die einzelnen Unterseiten analysierst? Auf der ersten Seite, der Dashboard Seite, gibt es ja noch keine wirklichen Daten zu analysieren, keine Trends zu erkennen, oder?\\\\
BM:\llabel{appendix:interview_trends_mundl:dashboard_1}  Genau, da komme ich erstmal an und verschaffe mir einen Überblick. Damit ich weiß, wie viele Kilometer haben wir erfasst, mit welchen Autos, welche Autos sind überhaupt erstmal in der Analyse verfügbar. Einfach ein Gefühl, dafür kriegen, welche Datenbasis vorliegt. \\\\
I: Und bei der Überwachung Dauerlauferprobung?\\\\
BM:\llabel{appendix:interview_trends_mundl:dauerlauferprobung_1}\llabel{appendix:interview_trends_mundl:fahrer_1} Da wird es dann natürlich schon wichtig. Die Dauerlauferprobung soll eigentlich immer dasselbe sein. Also blöd gesagt, idealerweise würde man diese Dauerlauferprobung auf einem Prüfstand fahren, was aber realistisch nicht möglich ist, weil das technisch nicht funktioniert. Das wurde schon ausgetestet, aber man war nie erfolgreich. Also nicht mal nur monalysis, sondern wirklich große Firmen und Tariffunktionäre. Oder das sollte von Robotern auf dem Erprobungsgelände gefahren werden. Die tatsächlich immer gleich fahren. Beispielsweise die erste Strecke immer mit 25 km/h, die zweite Stecke immer mit 12,3 km/h. Geht theoretisch, ist aber natürlich schon eher teurer, bis mal so ein kompletter Roboterbetrieb auf einem Testgelände realisiert wurde. Besonders schwierig ist das Ganze, wenn der Roboterbetrieb mit verschiedenen Fahrzeugen betrieben werden soll. Von dem her wurde das von den Firmen noch nicht in der Gänze realisiert.\\ Also werden die Fahrzeuge von Fahrern gefahren. Das ist die aktuell günstigste Alternative.\llabel{appendix:interview_trends_mundl:fehler_1} Da ist allerdings das Problem, dass wenn ein Fahrer mehrere Wochen lang immer die gleiche Strecke fährt, einfach Inkonsistenzen dabei auftreten. Oder er kennt die Strecke bereits so gut, dass es genau weiß, wo beispielsweise Schlaglöcher sind und umfährt diese, um nicht durchgeschüttelt zu werden. Das ist problematisch, da die ganze Dauerlauferprobung dann inkonsistent wird. Das bedeutet, dass die gefahrene Dauerlauferprobung dann nicht mehr zu den Berechnungsgrundlagen passt, dass die Annahmen, wenn X km diese Erprobung gefahren wird, dann entspricht das dem Fahrzeug tatsächlich.\llabel{appendix:interview_trends_mundl:dauerlauferprobung_2} Das muss das Fahrzeug widerspiegeln, sonst gibt es keine Aussage, ob die Referenz, die gefahren wurde, überhaupt das widerspiegelt, was berechnet wurde. Deswegen ist es wichtig, die Dauerlauferprobung zu kennen, damit man kontrollieren kann, inwiefern die Strecke immer gleich gefahren wurde, und damit man direkt schon erkennen kann, ob Schäden aufkommen, weil wenn nachvollzogen werden kann, dass die Fahrer die Fahrzeuge immer gleich gefahren haben, aber der Verlauf der Dauerlaufüberwachung ändert sich plötzlich, dann ist das problematisch.\llabel{appendix:interview_trends_mundl:schädigungswerte_1}\llabel{appendix:interview_trends_mundl:dauerlauferprobung_3} Der Verlauf der Dauerlaufüberwachung wird widergespiegelt durch Schädigungswerte, das heißt für jede Strecke wird ein Schädigungswert ausgerechnet.\llabel{appendix:interview_trends_mundl:schädigungswerte_2} Beispielsweise, wenn der Belgisch Block korrekt befahren wird, dann ist Zahl X der standardmäßige Schädigungswert. Diese Werte kann man dann aufsummieren und anhand dessen kann man die Werte visualisieren.\llabel{appendix:interview_trends_mundl:aufstiege_1}\llabel{appendix:interview_trends_mundl:defekte_3} Wenn die Werte dann einen riesen Sprung darstellen, dann ist irgendwo ein Problem aufgetreten, das kann dann auch daran liegen, dass das Fahrzeug einen Defekt hat, weil man fährt diese Erprobung ja eigentlich nur, um zu wissen, ob das Fahrzeug das aushält. Also sind Schäden irgendwo vorprogrammiert. Es ist also keine Besonderheit, dass während der Erprobung mal ein Teil kaputtgeht. Es geht darum, an, die Grenze des Fahrzeugs zu kommen.\\ Und von dem her sind diese beiden Aspekte das Wichtige. Zum einen das Überwachen, wurde alles immer gleich gefahren. Und zum anderen können jetzt schon Schäden erkannt werden.\\\\ 
I: Vielen Dank, also so Hüpfer oder Anomalien ist ja das, wo Trends potenziell erkannt werden können.\\\\
BM:\llabel{appendix:interview_trends_mundl:defekte_4}\llabel{appendix:interview_trends_mundl:fehler_2}\llabel{appendix:interview_trends_mundl:fahrer_2} Ja, die können aber zwei Ursachen haben, schlechtes Fahren oder Defekte.\\\\
I: Die Ursachen sind erstmal gar nicht so wichtig, wichtig ist zu erkennen, da passiert etwas mit den Daten und das ist nicht ganz normal und das dementsprechend darzustellen.\\\\
BM:\llabel{appendix:interview_trends_mundl:abweichungen_1} Im Endeffekt, kann man schon bevor man losfährt, eine Prognose treffen, wie die Schädigung bei 10.000 km, 20.000 km und so weiter aussieht. Vom Gesamtschädigungswert. Und wenn das vom Soll abweicht, dann kommt es zu diesen Anomalien.\\\\
I: Okay, und dann noch zum Vergleich von Einsatzdaten, das ist die letzte Seite.\\\\
BM: Ja, das ist dann die eigentliche Analyse selber.\llabel{appendix:interview_trends_mundl:dauerlauferprobung_4} Weil da ist dann die Dauerlauferprobung geführt und das hat alles funktioniert. Und dann wissen wir, okay, das, was bei der Dauerlauferprobung gefahren wurde, das nennen wir Zyklus, es kann auch Erprobungsmaßstab genannt werden. Dieser Zyklus bedeutet, bis X Kilometern ist das Fahrzeug erprobt. Was danach passiert, das wissen wir nicht. Und dieser Zyklus wird dann auf 100 \% Belastung gesetzt. In der Betriebsfestigkeit ist das dann nicht 100 \%, sondern eine 1.\llabel{appendix:interview_trends_mundl:vergleich_1} Und deswegen werden dann die Schädigungswerte auf die 1 gelegt. Danach kann der Erprobungszyklus gespiegelt werden und für andere Strecken eingesetzt werden. Und diese Daten können dann gebündelt werden, um die neuen Einsätze im Vergleich zum Erprobungszyklus zu analysieren. Dadurch kann man die Schädigungen hoch multiplizieren, um eine Zahl zu berechnen, bei der die Schädigung des Erprobungszyklus erreicht ist. So kann entschieden werden, welche Einsätze relevant sind, und welche Komponenten werden, wann stark belastet, um beispielsweise Ersatzteile mit auf den Einsatz zu nehmen. So lässt sich auch eine ungefähre Aussage treffen, wann welche Komponenten versagen. Also soll in der Darstellung gezeigt werden, wie der gezeigte Einsatz im Vergleich zu der Auslegung, also der Dauerlauferprobung ist. Hierbei sollen aber auch andere Einsätze miteinander verglichen werden können. Die Frage ist, ob es dann auch möglich ist, zu sagen, wenn der Kunde auf diesen Einsatz aufbricht, sollte er dann bestimmte Ersatzteile mitnehmen, weil mit dem Ausfall von diesen gerechnet werden kann. Hier geht es dann richtig in das ingenieursmäßige Arbeiten rein, dass man lernen und verstehen muss, was man aus den Daten herausziehen kann. Das heißt, diese letzte Seite ist das Spannendste.\llabel{appendix:interview_trends_mundl:vergleich_2} Da können dann auch richtige Querverweise zu älteren Einsätzen getroffen werden. Dass zum Beispiel bei Neuentwicklungen bei dem Bauteil XY die Schädigung YZ vorgekommen ist, dann können auch beispielsweise ältere Daten bei ähnlichen Bauteilen verglichen werden, ob die Werte plausibel sind. Da kann es sogar sinnvoll sein, eine Einzelteilanalyse zu starten. \\\\
I: Sehr gut, sehr spannend. Was würdest du dir wünschen, wenn du dir eine Visualisierung anschaust?\\\\
BM:\llabel{appendix:interview_trends_mundl:extremwerte_2} Was ich unbedingt brauche, ist die minimal und maximale Schädigung, also die Extremwerte, ob als Linie oder als Balken, das ist mir egal. Aber auf jeden Fall als Diagramm, als Zahl ist das schwer einzuordnen, zu häufig zu viele Zahlen.\llabel{appendix:interview_trends_mundl:schädigungswerte_3} Was auch immer sehr, sehr hilfreich ist, sind Schädigungswerte, das hilft einfach enorm für das grobe Einschätzen von verschiedenen Messungen. Der sagt zwar nicht alles aus, so ein Schädigungswert, aber er ist sehr hilfreich, dass man für den Kontext der Messung einfach einen Überblick hast, ist das allgemein ein härterer oder weicherer Untergrund.\llabel{appendix:interview_trends_mundl:zaehlmethoden_1} Dann die Kollektive, Rangepair, Levelcrossing, die sind einfach in der Betriebsfestigkeit oder Auslegung das Mittel, mit denen können wir Ingenieure sehr gut arbeiten. Oder auch manchmal ein Time at Level, das ist so ein ähnliches Kollektiv. Also eine Verweil Aufzählung, wie lang ist der Zustand vorhanden. Also auf der y-Achse die Amplituden und auf der x-Achse die Zeit in Sekunden zum Beispiel. Gerade so Drehzahlen sind dafür gut geeignet.\llabel{appendix:interview_trends_mundl:zeitreihen_1} Und natürlich, wenn etwas heraussticht in den Graphen, dann sollte eine detaillierte Ansicht vorhanden sein. Und dafür werden eigentlich die Zeitreihen genutzt, weil anhand von den Zeitschrieb kann man sehr gut erkennen, was wann passiert ist.\llabel{appendix:interview_trends_mundl:wunsch_1} Und da wäre es sehr, sehr hilfreich, wenn man verschiedene Zeitschriebe vergleichen könnte. Das ist dann aber ziemlich aufwändig, weil man dafür auch die Geschwindigkeit im Betracht ziehen muss. Ansonsten werden ja bei jedem Rad beispielsweise getrennt die Werte erfasst, und wenn aber die Kanäle sehr unterschiedlich sind, dann sollte man diese vergleichen können, um zu sehen, ob die Unterschiede noch in einem realistischen Bereich sind. Also so Querverweise zwischen verwandten Messstellen. Dann vielleicht noch die Geschwindigkeit und GPS, falls vorhanden. Oder die Querbeschleunigung, um an einem Rundkurs herzuleiten, an welcher Stelle man sich befindet. So etwas in Richtung Location. Damit lässt sich das Ganze gut zusammenfassen.\\\\
I: Okay, dann vielen Dank für deine Zeit und noch einen schönen Tag.\\\\
BM: Danke gleichfalls.
\end{linenumbers}
