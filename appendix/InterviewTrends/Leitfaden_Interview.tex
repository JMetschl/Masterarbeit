\chapter{Experteninterview zu Trends in den Daten}
\section{Interviewleitfaden}
\label{appendix:interview_trends}
\begin{description}
\item[Forschungsfrage]\hfill \\\\
Welche Trends sind in Bezug auf die Daten für den Kunden von Interesse?\\
\item[Einstieg]\hfill \\\\
Die Interviews dienen dazu, mithilfe der Experten zu erforschen, welche Trends in den Daten durch das Data Storytelling dargestellt werden sollen.
Kurzer Einstieg in das Thema der Masterarbeit und Erklärung der Forschungsfrage und dem Zweck des Interviews.\\
\item[Hauptteil]\hfill \\\\
Welche Daten werden von den Sensoren geliefert?\\
Worauf achten Sie, wenn Sie die Daten analysieren?\\
Im Bezug auf die bereits geplanten Unterseiten\\
\begin{itemize}
    \item Datenbasis\hfill \\Ein Dashboard, das einen Überblick über die vorhandenen Fahrzeuge und Flotten geben soll
    \item Überwachung Dauerlauferprobung\hfill \\Eine Seite zur Analyse von einer einzelnen Fahrt während der Dauerlauferprobung, also eine Fahrt auf einer abgeschlossenen Strecke, wobei die Strecke, sowie der Untergrund immer gleich bleiben
    \item Vergleich von Einsatzdaten\hfill \\Eine Seite, die zur Auswahl von zwei verschiedenen Eingaben genutzt werden kann, um diese zu vergleichen\\
\end{itemize}
sind welche Visualisierungen oder Informationen in den Daten interessant?\\
\item[Rückblick und Zusammenfassung]\hfill \\\\
Insgesamt stimmen die Interviews, sowie die Workflows und Points of Interest aller interviewten Personen zum größten Teil überein. Daraus lässt sich zusammenfassen, dass für die Trends Aufstiege, Abstiege, Extremwerte, Offsets, Schwingungen und allgemeine Abweichungen hervorgehoben werden können. Ebenso können Defekte und Messfehler oder Fehler des Fahrers gekennzeichnet werden.

\end{description}