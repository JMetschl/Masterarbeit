\section{Transkription Interview Michael Städele}
\label{appendix:interview_trends_staedele}
Interviewpartner: Diplomingenieur (FH) Michael Städele, seit 2008 in der Messdatenanalyse mit Fahrzeugen und Flugzeugen mit Betriebsfestigkeit\\
Datum: 16. November 2023, um 9 Uhr\\
Dauer des Interviews: 15 Minuten\\

\begin{linenumbers}
\noindent
I: Hallo, Herr Städele, vielen Dank, dass du dir für das Interview Zeit nimmst. Wie du ja schon weißt, geht es bei dem Thema meiner Masterarbeit um das Frontend des neuen Kundenprojekts. Und zwar, darum, dass man die Graphen mit Data Storytelling darstellt. \\\\
MS: Ja, kein Problem.\\\\
I: So, beim Data Storytelling ist es ja so, das ist eine Art, wie Daten dargestellt werden können. Dass aus den Daten eine Geschichte erzählt wird und euch, wenn ihr Daten auswertet, die Arbeit erleichtern wird. Soweit das möglich ist. Und meine Forschungsfrage ist in dem Sinne, welche Trends es in den Daten gibt, die man erkennen könnte, um die darzustellen, um euch zu helfen.\\\\
MS: Welche Trends?\\\\
I: Genau, also sowas wie hier ist die Schädigung hochgegangen oder heruntergegangen, und damit ich das gleich hervorheben kann, damit ihr gleich seht, was für euch spannend ist.\\\\
MS: Also rein auf Basis erstmal von der erfassten Daten?\\\\
I: Von den erfassten Daten, genau. Und rein darauf ausgelegt, was ihr interessant findet beim Auswerten.\\\\
MS:\llabel{appendix:interview_trends_staedele:aufstiege_1}\llabel{appendix:interview_trends_staedele:schädigungswerte_1} Also rein in den Daten, wenn man sich die anschaut, da ist ja immer, wenn eine hohe Amplitude ist, also wenn es hochgeht, dann ist die Schädigung immer relativ hoch.\llabel{appendix:interview_trends_staedele:abstiege_1} Wenn es niedrig ist, dann ist es immer im Verhältnis niedrig.\llabel{appendix:interview_trends_staedele:storytelling_1} Jetzt ist ja die Sache, die interessant ist, wie verhält sich der tatsächliche Einsatz, zu dem, was die Fahrzeughersteller oder eben auch der Kunde erprobt. Das heißt, einmal muss man irgendwie sehen, was macht der Kunde auf seinem Testgelände, Schädigungs-mäßig, und wie verhält sich das Ganze zu dem tatsächlichen Einsatz. Und wo im Einsatz ist es extrem hart, also zum Beispiel in welche Regionen, aber auch vielleicht welche Fahrweisen, welche besonderen Vorkommnisse könnten da einen Einfluss haben. Also ich sage jetzt einfach mal in dem Fall des Kundens, wenn die Gefahrenlage relativ gering ist, dann denke ich, wird da nicht so viel passieren, weil man dann sehr gemütlich fahren kann, wenn jetzt die Gefahrenlage sehr hoch ist, könnte es auch sein, dass auch auf Strecken, die vielleicht nicht so ganz schlimm sind, trotzdem hohe Belastungen auftreten können, weil der Fahrer übelst Gas geben muss. Was könnten die Daten noch so erzählen?\llabel{appendix:interview_trends_staedele:storytelling_2} Wie denn auch die Fahrbahn beschaffen ist, das heißt, wenn ich auf einem guten Asphalt fahre, dann wird da natürlich nicht so viel passieren, wie wenn ich auf einem Schotterweg fahre. Oder irgendeinem tiefen Gelände. Auch wenn da die Geschwindigkeiten relativ gering sind. Also kann man auch, je nach Region, irgendwo Rückschlüsse ziehen auf die Topologie von dem Land. Und dann ist auch noch eine Sache, die sehr interessant sein könnte, wie verhalten sich verschiedene Einsatzzwecke untereinander. Beispielsweise ganz normale Einsatzfahrzeuge und Fahrschulfahrzeuge. Ja, das sind so Sachen, die mir so auf die Schnelle einfallen, die solche Daten erzählen könnten.\\\\
I: Sehr gut, ich habe auch noch ein paar andere Fragen mitgebracht.\\\\
MS: Okay.\\\\
I: Welche Daten werden von den Sensoren genau geliefert?\\\\
MS:\llabel{appendix:interview_trends_staedele:daten_1} Also von den Sensoren im Fahrzeug werden geliefert, also die wir da zusätzlich erfassen, sind im Endeffekt Beschleunigungssignale im Fahrzeuginnenraum, in den drei Richtungen, also x, y, z und ich glaube, es sind auch noch die drei rotatorischen, also die Momente um x, y, z mit dabei. Und ansonsten haben wir wahrscheinlich noch Zugriff auf Fahrzeug-interne Sensorik, relativ einfache, ich kann aktuell noch nicht genau sagen, welche, ob dann da Lenkwinkel, oder eine Bremspedalstellung, oder ein Gang oder eine Geschwindigkeit, dabei sind. Und wünschenswert wäre noch GPS, aber das weiß ich noch nicht, ob das möglich ist. Das ist so das was geliefert wird und im Erprobungsprozess, also wenn die Fahrzeuge auf dem Testgelände fahren, werden meistens noch \ac{DMS} mit angeschlossen. Also wirklich, um dann die Kräfte an den einzelnen Komponenten zu messen. Und das ist dann der Zweck mit dem \ac{DT}, dass man das später nicht mehr messen muss, sondern nur noch an den anhand von den Beschleunigungen ableiten kann.\\\\
I: Alles klar, worauf achtest du, wenn du die Daten auswertest? Gibt es da einen Workflow, oder Points of Interest in den Graphen, die besonders spannend sind?\\\\
MS:\llabel{appendix:interview_trends_staedele:defekte_1} Also das allererste, was wichtig ist, ist überhaupt mal die Plausibilisierung von den Daten. Die ganze Messtechnik, die schon noch ganz schön fehleranfällig. Das ist nicht nur bei uns so, sondern generell, dass da einfach irgendwelche kurzen Sensorfehler oder irgendwelche Pikser drin sein können, das heißt es ist wichtig, dass die da eigentlich raus sind. Dass man die vorab erkennt, bevor man die wirkliche Auswertung macht.\llabel{appendix:interview_trends_staedele:aufstiege_2}\llabel{appendix:interview_trends_staedele:peaks_1} Und wenn ich jetzt einfach eine Zeitreihe anschaue, dann sind immer interessant, eigentlich hohe Ausschläge. Also gerade, wenn irgendwo recht hohe Belastungen auftreten, dann ist da ja irgendwo erstmal was Größeres passiert, als beim normalen Fahren.\llabel{appendix:interview_trends_staedele:schwingungen_1} Und was auch noch so ein Punkt ist, ist so ein bisschen so eine Trennung von dem Höher und Niederfrequenten Zeug, von so Fahrmanöver-Geschichten und auch von höherfrequenteren Sachen, die dann eher durch die Fahrbahnunebenheit hervorgerufen werden. Ja, und sonst, rein so optisch, wenn ich mir irgendwas anschaue, dann schaut man ja immer auf was, was markiert ist. Wenn ich mir eine Strecke anschaue, dann schaue ich ja auf Daten, die mit einer Signalfarbe 
hervorgehoben werden. \\\\
I: Genau, das versuche ich ja herauszufinden, was ich am besten hervorhebe. Beziehungsweise zu was für einer Konklusion ich auch kommen kann durch die Daten, die ich automatisch berechnen lassen kann, in dem ich mir so Trends überlege. \\\\
MS:\llabel{appendix:interview_trends_staedele:schädigungswerte_2} Was wir ja natürlich auch machen, sind den Zeitreihen ja die Schädigung abzuleiten. Dann gibt es ja zum Beispiel auch so Schädigungsevolutionen. Also einfach, wo dann über der Strecke oder der gefahrenen Zeit dann die Schädigung aufgetragen ist. Und da ist ja auch immer, wenn es irgendwo hochgeht, immer sehr interessant.\\\\
I: Also das wäre so ein Punkt, den du gerne hervorgehoben hättest?\\\\
MS:\llabel{appendix:interview_trends_staedele:peaks_2} Ja, ja, oder das ist die Frage wie man es dann zusammen bringt, mit einer Karte, oder? Das man sieht okay und hier ist es gerade besonders. Aber vom Prinzip her, wenn irgendwo was Höheres ist, dann ist es da interessant. \\\\
I: Also höher als jetzt der Durchschnitt?\\\\
MS: Als der Durchschnittswert zum Beispiel, genau. \\\\
I: Alles klar. Gut, und wir haben ja schon Unterseiten festgelegt, für die Webseite. Und da würde mich interessieren, was für Visualisierungen und/oder Informationen allgemein besonders spannend  wäre, für dich als auswertende Person. Und zwar haben wir ja die Datenbasis, da ist es ja an sich so, dass es noch gar keine Visualisierungen gibt, das ist ja nur so ein Überblick.\\\\
MS: Genau.\\\\
I: Das hatte ich auch so weiter geplant, dass es da noch keine Graphen gibt.\\\\
MS: Aber zum Beispiel in der Datenbasis, ich weiß nicht genau, was du unter Visualisierungen verstehst, aber die Fahrzeugbilder und sowas sind ja dann da auch drin, oder?\\\\
I: Ja, ich rede von Graphen.\\\\
MS: Ah, okay.\\\\
I: Dann die Überwachung Dauerlauferprobung, da wird es ja dann schon spannender.\\\\
MS: Was ich da gerne sehen würde? In der Überwachung Dauerlauferprobung.\\\\
I: Ja, genau.\\\\
MS:\llabel{appendix:interview_trends_staedele:dauerlauferprobung_1}  Also da würde ich mir zum Beispiel wünschen, dass man die verschiedenen Erprobungsstrecken nach Überfahrt miteinander vergleichen kann. Also wenn ich die gestern 10-mal gefahren bin und heute 7-mal, wie waren die Überfahrten? Waren die recht ähnlich, oder sind die sehr unterschiedlich. Dann würde mich interessieren, nicht nur von der Belastung her, sondern auch von der Fahrweise. Fahren die Fahrer mit den vorgegebenen Geschwindigkeiten, oder machen die da irgendwas? Zum Beispiel. \\\\
I: Also dann so ein klassischer Vergleich?\\\\
MS: Ja genau.\\\\
I: Aber wir haben ja noch die zweite Seite, die Vergleich von Einsatzdaten darstellt. Wäre das dann nicht so etwas Ähnliches?  Oder ist die Überwachung Dauerlauferprobung einfach immer die gleiche Strecke, die verglichen wird?\\\\
MS:\llabel{appendix:interview_trends_staedele:dauerlauferprobung_2} Also bei der Dauerlauferprobung ist es so, die fahren auf dem Testgelände und da haben sie verschiedene Strecken. Und dann gibt es eigentlich einen vorgegebenen Zyklus. Und diesen Zyklus sollen die während ihrer Erprobung fahren. Immer wieder. Und dann schaut man während der Erprobung, treten irgendwelche Schwachstellen auf, also geht irgendwo etwas kaputt? Und wenn was kaputtgeht, muss man das irgendwie in Verbindung bringen mit dem, was ich gefahren bin. Das heißt, deswegen wird dann da auch die Belastung mit Sensorik erfasst. Und da gibt es unterschiedliche Strecken, und die Strecken regen unterschiedliche Fahrzeugkomponenten an, einfach auf Basis ihrer Geometrie. Also der Sinus zum Beispiel in einer festen Frequenz oder der belgisch Block dann in allen Frequenzen. Und so ein Feldwaldweg oder so etwas, also so eher natürliche Strecken, die so auch wirklich vorkommen können. Und da vergleicht man ja, erst mal die Strecken untereinander, was bringt jede Strecke in das Fahrzeug ein? Und am Ende habe ich das erprobt, was ich erproben wollte? Also im Endeffekt mache ich ja am Anfang einmal meine Messungen und das ist dann meine Referenz und dann vergleiche ich über die paar tausend Kilometer, die die dann fahren, ist das auch eingebracht worden, was ich einbringen wollte?\llabel{appendix:interview_trends_staedele:vergleich_1} Das heißt, dann kann ich vergleichen und auch wirklich schauen, ob meine Erprobung korrekt war, und das eingebracht hat, was ich wollte. Oder auch nicht. Das ist bei der Erprobung. Und dann habe ich im Einsatz einen Vergleich davon, was tatsächlich passiert. Also was dann in tatsächlichen Ländern oder auf den tatsächlichen Strecken passiert im Vergleich, zu dem, was erprobt wurde. Und auch, wie sind die Regionen, die Strecken, die Länder untereinander? Ist das Streckengebiet viel härter als das andere oder da ist es deutlich weicher, weil die bessere Straßen haben. Und woanders ist es deutlich härter. Und in Kombination mit dem Wissen, was erprobt wurde und wann irgendwelche Komponentenausfälle waren, kann man abschätzen, wenn das Fahrzeug irgendwo fährt. Wie hoch ist die Wahrscheinlichkeit, dass irgendetwas ausfällt oder nicht. Genau.\\\\
I: Okay.\\\\
MS:\llabel{appendix:interview_trends_staedele:storytelling_3} Oder was fällt mir noch dazu ein? Oder wie nutzen denn die Fahrer tatsächlich das Fahrzeug? Wie lang ist das Fahrzeug in welchem Geschwindigkeitsbereich? Oder steht das nur herum?\\\\
I: Okay, sehr gut, dann haben wir es auch schon geschafft.\\\\
MS: Ach, cool.\\\\
I: Vielen Dank für deine Zeit und die ausführlichen Erklärungen.\\\\
MS: Kein Problem, jederzeit gerne wieder.
\end{linenumbers}