%*******************************************************
% Abstract in German
%*******************************************************
\begin{otherlanguage}{ngerman}
	\pdfbookmark[0]{Zusammenfassung}{Zusammenfassung}
	\chapter*{Zusammenfassung}
Diese Masterarbeit untersucht die Anwendung von Data-Storytelling auf kraftfahrzeugbasierte Daten. Dabei wird das Potenzial für eine Vereinfachung der abstrakten Daten zugunsten einer effektiven Kommunikation betrachtet. Durch die Ergebnisse einer Experteninterviewreihe wurden Einblicke in die Erfahrungen von Fachleuten der Betriebsfestigkeit gewonnen, um die Herausforderungen bei der Interpretation und Darstellung solcher komplexen Informationen zu verstehen.\\\\
Die Arbeit präsentiert eine Reihe von implementierten Graphen, die auf den gewonnenen Erkenntnissen basieren und Data-Storytelling-Techniken verwenden, um die Daten in eine verständliche Form zu bringen. Diese Visualisierungen haben zum Ziel, komplizierte Informationen zu vereinfachen und dabei die wichtigsten Aspekte in den Daten hervorzuheben. Zusätzlich werden Muster und auch Trends analysiert.\\\\
Ein weiterer Schwerpunkt der Arbeit liegt in der Trenderkennung, die speziell auf die Bedürfnisse der Zeitreihen zugeschnitten ist, die aus den kraftfahrzeugbasierten Daten gewonnen werden. Durch die Implementierung einer dynamischen Trenderkennung in TypeScript werden Muster und Entwicklungen in den Zeitreihendaten automatisch erkannt. Dabei wurde durch die Implementierung als Webseite besonders auch auf eine flüssige Performance, durch Einbezug aller \ac{PC}-Komponenten, geachtet. \\\\
Zusammenfassend trägt die Arbeit dazu bei, das Verständnis der Anwendung von Data-Storytelling auf kraftfahrzeugbasierte Daten zu vertiefen, und bietet praktische Einblicke in die Umsetzung von Trenderkennungsalgorithmen in TypeScript. Die Ergebnisse und Erkenntnisse leisten einen wertvollen Beitrag für Fachleute in der Automobilindustrie und Betriebsfestigkeitsbranche sowie für Forscher im Bereich Datenvisualisierung und -analyse.
\end{otherlanguage}
